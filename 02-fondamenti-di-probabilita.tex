\input{preambolo_comune}

% --- Titolo ---
\title{Fondamenti di Probabilità}
\author{Basato sulle prove d'esame}
\date{\today}

\begin{document}

\maketitle
\tableofcontents
\newpage

\label{cap:fondamenti}
In questo capitolo introdurremo i concetti di base del calcolo delle probabilità, che sono il mattone fondamentale per tutto ciò che seguirà.

\section{Spazio Campionario (\texorpdfstring{$\Omega$}{Omega}) ed Eventi}

\begin{definition}[Esperimento Aleatorio]
Un \textbf{esperimento aleatorio} è un processo il cui esito non può essere previsto con certezza prima che venga eseguito, anche se tutte le possibili realizzazioni sono note.
\end{definition}

\begin{example}
Esempi di esperimenti aleatori sono:
\begin{itemize}
    \item Il lancio di una moneta (possibili esiti: Testa, Croce).
    \item Il lancio di un dado a sei facce (possibili esiti: 1, 2, 3, 4, 5, 6).
    \item L'estrazione di una carta da un mazzo.
    \item La scelta di uno studente da una classe per un'interrogazione.
\end{itemize}
\end{example}

\begin{definition}[Spazio Campionario \texorpdfstring{$\Omega$}{Omega}]
Lo \textbf{spazio campionario}, indicato con $\Omega$, è l'insieme di tutti i possibili esiti (o risultati elementari) di un esperimento aleatorio.
\end{definition}

\begin{example}
\begin{enumerate}
    \item Lancio di una moneta: $\Omega = \{T, C\}$ (dove T=Testa, C=Croce).
    \item Lancio di un dado: $\Omega = \{1, 2, 3, 4, 5, 6\}$.
    \item Lancio di due monete (o una moneta due volte): $\Omega = \{(T,T), (T,C), (C,T), (C,C)\}$. Nota che l'ordine può essere importante.
    \item \textbf{Esercizio 2, 08/09/2023 (Scelta gruppo):} Una classe di 10 studenti viene suddivisa in due sottogruppi di 5. Il primo va a Firenze, il secondo a Roma. Se definiamo un esito come la composizione del gruppo che va a Roma (o Firenze), $\Omega$ è l'insieme di tutte le possibili combinazioni di 5 studenti scelti da 10. Il numero di tali esiti è $\binom{10}{5}$.
\end{enumerate}
\end{example}

\begin{definition}[Evento]
Un \textbf{evento} $A$ è un qualsiasi sottoinsieme dello spazio campionario $\Omega$ (cioè $A \subseteq \Omega$). Un evento si dice \textbf{verificato} (o che si è realizzato) se l'esito dell'esperimento aleatorio è un elemento di $A$.
\end{definition}

\begin{example}
Consideriamo il lancio di un dado, $\Omega = \{1, 2, 3, 4, 5, 6\}$.
\begin{itemize}
    \item Evento $A = \text{"esce un numero pari"}$: $A = \{2, 4, 6\}$.
    \item Evento $B = \text{"esce un numero maggiore di 4"}$: $B = \{5, 6\}$.
    \item Evento $C = \text{"esce il numero 3"}$: $C = \{3\}$ (questo è un evento elementare).
    \item L'insieme vuoto $\emptyset$ è un evento, chiamato \textbf{evento impossibile} (non può mai verificarsi).
    \item Lo spazio campionario $\Omega$ è un evento, chiamato \textbf{evento certo} (si verifica sempre).
\end{itemize}
\end{example}

\subsection{Operazioni tra Eventi}
Essendo gli eventi degli insiemi, possiamo applicare le usuali operazioni insiemistiche:
\begin{itemize}
    \item \textbf{Unione ($A \cup B$):} L'evento "si verifica $A$ oppure si verifica $B$ (o entrambi)". Corrisponde al connettivo logico "OR".
    \item \textbf{Intersezione ($A \cap B$):} L'evento "si verificano sia $A$ che $B$". Corrisponde al connettivo logico "AND".
    \item \textbf{Complementare ($A^c$ o $\bar{A}$):} L'evento "non si verifica $A$".
\end{itemize}
Due eventi $A$ e $B$ si dicono \textbf{incompatibili} (o \textbf{disgiunti} o \textbf{mutuamente esclusivi}) se non possono verificarsi contemporaneamente, cioè se $A \cap B = \emptyset$.

\begin{example}
Sempre con il lancio del dado:
\begin{itemize}
    \item $A = \{2, 4, 6\}$, $B = \{5, 6\}$.
    \item $A \cup B = \{2, 4, 5, 6\}$ (esce un numero pari OPPURE un numero maggiore di 4).
    \item $A \cap B = \{6\}$ (esce un numero pari E maggiore di 4).
    \item $A^c = \{1, 3, 5\}$ (non esce un numero pari, cioè esce un numero dispari).
    \item Se $D = \{1, 3\}$, allora $A \cap D = \emptyset$, quindi $A$ e $D$ sono incompatibili.
\end{itemize}
\end{example}

\section{Definizione di Probabilità}
Vogliamo assegnare un numero a ciascun evento che ne misuri la "possibilità" di verificarsi.

\begin{definition}[Probabilità]
Dato uno spazio campionario $\Omega$, una \textbf{funzione di probabilità} $\Prob$ è una funzione che assegna a ogni evento $A \subseteq \Omega$ un numero reale $\Prob(A)$ tale che siano soddisfatti i seguenti assiomi (assiomi di Kolmogorov):
\begin{enumerate}
    \item $\Prob(A) \ge 0$ per ogni evento $A$ (non negatività).
    \item $\Prob(\Omega) = 1$ (normalizzazione: l'evento certo ha probabilità 1).
    \item Se $A_1, A_2, \dots, A_n, \dots$ è una successione di eventi a due a due incompatibili (cioè $A_i \cap A_j = \emptyset$ per $i \neq j$), allora
    \[ \Prob\left(\bigcup_{i=1}^{\infty} A_i\right) = \sum_{i=1}^{\infty} \Prob(A_i) \quad (\sigma\text{-additività}) \]
    Se lo spazio campionario $\Omega$ è finito, la $\sigma$-additività si riduce alla \textbf{additività finita}: se $A$ e $B$ sono incompatibili ($A \cap B = \emptyset$), allora $\Prob(A \cup B) = \Prob(A) + \Prob(B)$.
\end{enumerate}
\end{definition}

Da questi assiomi derivano alcune proprietà importanti:
\begin{itemize}
    \item $\Prob(\emptyset) = 0$.
    \item Se $A \subseteq B$, allora $\Prob(A) \le \Prob(B)$.
    \item $0 \le \Prob(A) \le 1$ per ogni evento $A$.
    \item $\Prob(A^c) = 1 - \Prob(A)$.
    \item $\Prob(A \cup B) = \Prob(A) + \Prob(B) - \Prob(A \cap B)$ (principio di inclusione-esclusione per due eventi).
\end{itemize}

\subsection{Spazi Equiprobabili e Calcolo Combinatorio}
In molti problemi, specialmente quelli che coinvolgono dadi, monete "bilanciate", o estrazioni "casuali", si assume che tutti gli esiti elementari in $\Omega$ abbiano la stessa probabilità di verificarsi. Questo è il caso di uno \textbf{spazio campionario equiprobabile} (o uniforme).

Se $\Omega$ è uno spazio campionario finito con $N = |\Omega|$ esiti elementari, e se ogni esito ha la stessa probabilità, allora la probabilità di un singolo esito $\omega_i$ è $\Prob(\{\omega_i\}) = 1/N$.
In questo caso, la probabilità di un evento $A$ è data da:
\[ \Prob(A) = \frac{\text{numero di esiti favorevoli ad } A}{\text{numero di esiti possibili}} = \frac{|A|}{|\Omega|} \]

Il \textbf{calcolo combinatorio} ci fornisce gli strumenti per contare efficacemente $|A|$ e $|\Omega|$. La chiave è capire quale strumento usare in base alle caratteristiche del problema:
\begin{itemize}
    \item Se l'ordine conta e possiamo riutilizzare gli elementi → Disposizioni con Ripetizione
    \item Se l'ordine conta e NON possiamo riutilizzare gli elementi → Disposizioni Semplici
    \item Se l'ordine NON conta e NON possiamo riutilizzare gli elementi → Combinazioni Semplici
    \item Se dobbiamo solo ordinare tutti gli elementi → Permutazioni Semplici
\end{itemize}

\subsubsection{Principio Fondamentale del Conteggio (Metodo delle Scelte Successive)}
\label{subsubsec:principio_moltiplicazione}

Molti problemi di conteggio, specialmente quelli che coinvolgono la determinazione del numero di esiti possibili ($|\Omega|$) o del numero di esiti favorevoli a un evento ($|A|$), possono essere risolti scomponendo il processo di selezione o costruzione in una sequenza di \textbf{scelte successive}.

\noindent Il \textbf{Principio di Moltiplicazione} (o Regola del Prodotto) afferma che se un'operazione può essere eseguita in $k$ passi (o scelte), e
\begin{itemize}
    \item il primo passo può essere eseguito in $n_1$ modi,
    \item per ciascuno di questi modi, il secondo passo può essere eseguito in $n_2$ modi,
    \item per ciascuna combinazione dei primi due passi, il terzo passo può essere eseguito in $n_3$ modi,
    \item e così via, fino al $k$-esimo passo che può essere eseguito in $n_k$ modi (dove il numero di modi per una scelta può dipendere dalle scelte precedenti, ma è fissato una volta che tali scelte sono note),
\end{itemize}
allora il numero totale di modi per eseguire l'intera operazione è il prodotto $N = n_1 \cdot n_2 \cdot n_3 \cdot \dots \cdot n_k$.

\begin{example}[Scelta di un pasto]
Un ristorante offre un menù fisso con 3 antipasti, 4 primi e 2 dolci. Quanti pasti completi diversi si possono comporre scegliendo un antipasto, un primo e un dolce?
\begin{itemize}
    \item Scelta 1 (Antipasto): $n_1 = 3$ modi.
    \item Scelta 2 (Primo): $n_2 = 4$ modi (per ogni antipasto scelto).
    \item Scelta 3 (Dolce): $n_3 = 2$ modi (per ogni coppia antipasto-primo scelta).
\end{itemize}
Numero totale di pasti possibili: $N = n_1 \cdot n_2 \cdot n_3 = 3 \cdot 4 \cdot 2 = 24$.

\begin{figure}[h!]
    \centering
    \begin{tikzpicture}[
        node distance=1.2cm, % Distanza verticale tra i livelli
        every node/.style={
            align=center,
            text=primarytext,
            draw=nodecolor,
            fill=nodecolor!60, % Riempimento leggermente più chiaro
            rounded corners,
            minimum width=2.5cm,
            font=\small
        },
        edge from parent/.style={draw=linkcolor, thick, -latex}, % Stile per le frecce
        dots_node/.style={draw=none, fill=none, text=linkcolor!70, font=\Large} % Stile per i nodi "..."
    ]

    % Nodo radice (Scelta iniziale)
    \node (root) {Scelta del Pasto};

    % Livello 1: Antipasti
    % Mostriamo il primo antipasto e un nodo "..." per gli altri
    \node (ant1) [below=of root, xshift=-1.5cm] {Antipasto 1\\(\textbf{3} opzioni totali)};
    \node (ant_dots) [dots_node, below=of root, xshift=1.5cm] {$\vdots$};
    \draw (root) -- (ant1);
    \draw (root) -- (ant_dots); % Connessione per i puntini verticali

    % Livello 2: Primi per Antipasto 1
    % Mostriamo il primo primo e un nodo "..." per gli altri
    \node (primo1_ant1) [below=of ant1, xshift=-1cm] {Primo 1\\(\textbf{4} opzioni totali)};
    \node (primo_dots) [dots_node, below=of ant1, xshift=1cm] {$\vdots$};
    \draw (ant1) -- (primo1_ant1);
    \draw (ant1) -- (primo_dots);

    % Livello 3: Dolci per Primo 1
    % Mostriamo entrambi i dolci, dato che sono solo due
    \node (dolce1_primo1) [below=of primo1_ant1, xshift=-1.5cm] {Dolce 1\\(\textbf{2} opzioni totali)};
    \node (dolce2_primo1) [below=of primo1_ant1, xshift=1.5cm] {Dolce 2\\(\textbf{2} opzioni totali)};
    \draw (primo1_ant1) -- (dolce1_primo1);
    \draw (primo1_ant1) -- (dolce2_primo1);

    % Nodo per il calcolo finale
    \node (total_calc) [below=0.7cm of dolce1_primo1.south, xshift=2cm, text=primarytext, font=\large, draw=none, fill=none] {Totale combinazioni: \textbf{$3 \times 4 \times 2 = 24$}};

    \end{tikzpicture}
    \caption{Rappresentazione ad albero del metodo delle scelte successive per la composizione di un pasto.}
    \label{fig:albero_scelte}
\end{figure}
\end{example}

\begin{example}[Formare un codice]
Quanti codici di 3 caratteri si possono formare se il primo carattere deve essere una lettera maiuscola (26 possibilità), il secondo una cifra (10 possibilità) e il terzo una lettera minuscola (26 possibilità), senza ripetizioni per le lettere se fossero entrambe maiuscole o minuscole e dallo stesso gruppo? In questo caso, le scelte sono indipendenti:
\begin{itemize}
    \item Scelta 1 (Primo carattere, lettera maiuscola): $n_1 = 26$ modi.
    \item Scelta 2 (Secondo carattere, cifra): $n_2 = 10$ modi.
    \item Scelta 3 (Terzo carattere, lettera minuscola): $n_3 = 26$ modi.
\end{itemize}
Numero totale di codici: $N = 26 \cdot 10 \cdot 26 = 6760$.
Se, invece, i tre caratteri fossero tre cifre diverse da 0 a 9, avremmo: $n_1=10, n_2=9, n_3=8$, per un totale di $10 \cdot 9 \cdot 8 = 720$ codici. Questo illustra come $n_i$ possa dipendere dalle scelte precedenti.
\end{example}

Le formule per disposizioni e permutazioni possono essere viste come applicazioni dirette di questo principio. Ad esempio, per le Disposizioni Semplici $D_{n,k}$ (scegliere e ordinare $k$ elementi da $n$ senza ripetizione):
\begin{itemize}
    \item Per il primo elemento, ci sono $n$ scelte.
    \item Per il secondo elemento (diverso dal primo), ci sono $n-1$ scelte.
    \item ...
    \item Per il $k$-esimo elemento, ci sono $n-k+1$ scelte.
\end{itemize}
Il numero totale di disposizioni è $n \cdot (n-1) \cdot \dots \cdot (n-k+1)$.

Nell'esempio degli studenti biondi a Roma (Esempio~\ref{ex:studenti-roma}, discusso più avanti), quando si calcola il numero di modi per formare un gruppo con 2 biondi e 3 non biondi, $|B_2| = \binom{4}{2} \cdot \binom{6}{3}$, si sta usando il principio di moltiplicazione: la scelta dei 2 studenti biondi (tra 4 disponibili) e la scelta dei 3 studenti non biondi (tra 6 disponibili) sono due "passi" o "scelte" la cui numerosità di modi si moltiplica per ottenere il totale.

\paragraph{Collegamento con la Probabilità}
Quando gli "oggetti" che contiamo sono sequenze di eventi casuali, il principio di moltiplicazione per il conteggio si traduce, nel dominio delle probabilità, nella \textbf{Regola della Catena} (vedi Corollario~\ref{cor:catena}) per il calcolo della probabilità di un'intersezione di eventi. Questo aspetto è trattato in dettaglio nella Sezione~\ref{sec:prob_cond} sulla probabilità condizionata. Ad esempio, la probabilità di una sequenza di estrazioni da un'urna (come nell'esempio dell'Urna di Polya) è calcolata moltiplicando le probabilità condizionate di ogni estrazione, data la storia delle estrazioni precedenti.

\subsubsection{Disposizioni con Ripetizione}
\textbf{Formula:} $DR_{n,k} = n^k$

\textbf{Quando usarla:}
\begin{itemize}
    \item Quando possiamo \textbf{riutilizzare} gli stessi elementi
    \item Quando l'\textbf{ordine} è importante
    \item Quando dobbiamo fare $k$ scelte tra $n$ possibilità
\end{itemize}

\textbf{Esempi:}
\begin{enumerate}
    \item \textbf{Password di lunghezza 4 con cifre 0-9:}
    \begin{itemize}
        \item $n=10$ (cifre disponibili)
        \item $k=4$ (lunghezza password)
        \item Posso riusare le cifre (es: "1111" è valida)
        \item L'ordine conta ("1234" $\neq$ "4321")
        \item $DR_{10,4} = 10^4 = 10000$ possibili password
    \end{itemize}

    \item \textbf{Sequenze di 3 lanci di una moneta:}
    \begin{itemize}
        \item $n=2$ (T/C)
        \item $k=3$ (lanci)
        \item $DR_{2,3} = 2^3 = 8$ possibili sequenze: TTT, TTC, TCT, TCC, CTT, CTC, CCT, CCC
    \end{itemize}

    \item \textbf{Esempio:} 5 palline in 3 urne
    \begin{itemize}
        \item $n=3$ (urne A, B, C)
        \item $k=5$ (palline)
        \item Per ogni pallina posso scegliere una qualsiasi urna
        \item $DR_{3,5} = 3^5 = 243$ possibili distribuzioni
    \end{itemize}
\end{enumerate}

\subsubsection{Permutazioni Semplici}
\textbf{Formula:} $P_n = n!$

\textbf{Quando usarla:}
\begin{itemize}
    \item Quando dobbiamo ordinare \textbf{tutti} gli elementi di un insieme
    \item Quando ogni elemento deve essere usato \textbf{esattamente una volta}
    \item Quando l'\textbf{ordine} è importante
\end{itemize}

\textbf{Esempi:}
\begin{enumerate}
    \item \textbf{Anagrammi della parola "CASA":}
    \begin{itemize}
        \item $n=4$ (lettere)
        \item $P_4 = 4! = 24$ anagrammi
        \item CASA, ACSA, ASCA, SACA, CSAA, CASS, ...
    \end{itemize}

    \item \textbf{Ordinare 5 persone in fila:}
    \begin{itemize}
        \item $n=5$ (persone)
        \item $P_5 = 5! = 120$ possibili file
    \end{itemize}

    \item \textbf{Percorsi tra 4 città visitandole tutte:}
    \begin{itemize}
        \item $n=4$ (città)
        \item $P_4 = 4! = 24$ possibili percorsi
    \end{itemize}
\end{enumerate}

\subsubsection{Disposizioni Semplici}
\textbf{Formula:} $D_{n,k} = \frac{n!}{(n-k)!} = n \cdot (n-1) \cdot \dots \cdot (n-k+1)$

\textbf{Quando usarla:}
\begin{itemize}
    \item Quando NON possiamo riutilizzare gli elementi
    \item Quando l'\textbf{ordine} è importante
    \item Quando dobbiamo scegliere $k$ elementi da $n$
\end{itemize}

\textbf{Esempi:}
\begin{enumerate}
    \item \textbf{Podio in una gara con 10 atleti:}
    \begin{itemize}
        \item $n=10$ (atleti)
        \item $k=3$ (posizioni sul podio)
        \item L'ordine conta (1°, 2°, 3° posto)
        \item $D_{10,3} = \frac{10!}{7!} = 10 \cdot 9 \cdot 8 = 720$ possibili podi
    \end{itemize}

    \item \textbf{Scegliere presidente e tesoriere da 10 persone:}
    \begin{itemize}
        \item $n=10$ (persone)
        \item $k=2$ (ruoli)
        \item L'ordine conta perché i ruoli sono diversi (presidente $\neq$ tesoriere, quindi Alice presidente e Bob tesoriere è diverso da Bob presidente e Alice tesoriere)
        \item $D_{10,2} = 10 \cdot 9 = 90$ possibili assegnazioni
    \end{itemize}

    \item \textbf{PIN di 4 cifre diverse:}
    \begin{itemize}
        \item $n=10$ (cifre 0-9)
        \item $k=4$ (lunghezza PIN)
        \item Non posso riusare le cifre
        \item $D_{10,4} = 10 \cdot 9 \cdot 8 \cdot 7 = 5040$ possibili PIN
    \end{itemize}
\end{enumerate}

\subsubsection{Combinazioni Semplici}
\textbf{Formula:} $\binom{n}{k} = \frac{n!}{k!(n-k)!}$

\textbf{Quando usarla:}
\begin{itemize}
    \item Quando NON possiamo riutilizzare gli elementi
    \item Quando l'\textbf{ordine NON} è importante
    \item Quando dobbiamo scegliere $k$ elementi da $n$
\end{itemize}

\textbf{Esempi:}
\begin{enumerate}
    \item \textbf{Formare una squadra di 5 persone da un gruppo di 12:}
    \begin{itemize}
        \item $n=12$ (persone disponibili)
        \item $k=5$ (dimensione squadra)
        \item L'ordine non conta (è la stessa squadra)
        \item $\binom{12}{5} = \frac{12!}{5!(12-5)!} = 792$ possibili squadre
    \end{itemize}

    \item \textbf{Esempio:} Gruppo Roma/Firenze
    \begin{itemize}
        \item $n=10$ (studenti totali)
        \item $k=5$ (studenti per gruppo)
        \item L'ordine non conta
        \item $\binom{10}{5} = 252$ possibili gruppi
    \end{itemize}

    \item \textbf{Esempio:} Scegliere 2 lupi da 10 amici
    \begin{itemize}
        \item $n=10$ (amici)
        \item $k=2$ (lupi da scegliere)
        \item L'ordine non conta
        \item $\binom{10}{2} = 45$ possibili coppie di lupi
    \end{itemize}

    \item \textbf{Mano di 5 carte da un mazzo di 52:}
    \begin{itemize}
        \item $n=52$ (carte nel mazzo)
        \item $k=5$ (carte nella mano)
        \item L'ordine non conta
        \item $\binom{52}{5} = 2{,}598{,}960$ possibili mani
    \end{itemize}
\end{enumerate}

\begin{example}[Esercizio 2, 08/09/2023 - Studenti biondi a Roma]\label{ex:studenti-roma}
\textit{Problema:} Una classe di 10 studenti (4 biondi, 6 non biondi) è divisa in due gruppi di 5, uno per Roma e uno per Firenze. Qual è la probabilità che 2 studenti biondi vadano a Roma?

\textbf{Spiegazione Semplice:}
Immaginiamo di avere 10 studenti in classe:
\begin{itemize}
    \item 4 studenti biondi (chiamiamoli B1, B2, B3, B4)
    \item 6 studenti non biondi (chiamiamoli N1, N2, N3, N4, N5, N6)
\end{itemize}

\textbf{Cosa ci chiede il problema?}
Dobbiamo calcolare la probabilità che ESATTAMENTE 2 studenti biondi finiscano nel gruppo di Roma (che deve essere di 5 studenti in totale).

\textbf{Come si calcola la probabilità?}
Come sempre in probabilità, useremo la formula:
\[ \text{Probabilità} = \frac{\text{Casi favorevoli}}{\text{Casi possibili}} \]

\textit{Soluzione Passo Passo:}
\begin{enumerate}
    \item \textbf{Calcolare i casi possibili (denominatore):}
    \begin{itemize}
        \item Dobbiamo scegliere 5 studenti da un totale di 10 per mandarli a Roma
        \item L'ordine NON conta (non importa chi entra prima nel gruppo)
        \item Quindi usiamo le combinazioni: $\binom{10}{5}$
        \item $|\Omega| = \binom{10}{5} = \frac{10!}{5!5!} = \frac{10 \cdot 9 \cdot 8 \cdot 7 \cdot 6}{5 \cdot 4 \cdot 3 \cdot 2 \cdot 1} = 252$
    \end{itemize}

    \item \textbf{Calcolare i casi favorevoli (numeratore):}
    Qui sta il punto più difficile! Per avere ESATTAMENTE 2 biondi a Roma, dobbiamo:
    \begin{itemize}
        \item Scegliere 2 studenti biondi tra i 4 disponibili
        \item E ANCHE scegliere 3 studenti non biondi tra i 6 disponibili
        \item (perché il gruppo di Roma deve essere di 5 persone in totale!)
    \end{itemize}

    Usiamo il metodo delle scelte successive per calcolare il numero di casi favorevoli. \textbf{Perché?}
    Facciamo un esempio pratico:
    \begin{itemize}
        \item Prima scegliamo 2 biondi: potremmo scegliere B1 e B2
        \item Per OGNI scelta dei biondi, dobbiamo scegliere 3 non biondi: potremmo scegliere N1, N2, N3
        \item Quindi un possibile gruppo sarebbe: \{B1, B2, N1, N2, N3\}
        \item Ma potevamo anche scegliere B1 e B3 come biondi, e poi N1, N2, N3 come non biondi
        \item O B1 e B3 come biondi, e N2, N3, N4 come non biondi
        \item E così via...
    \end{itemize}

    Per ogni modo di scegliere 2 biondi ($\binom{4}{2} = 6$ modi), dobbiamo considerare tutti i modi di scegliere 3 non biondi ($\binom{6}{3} = 20$ modi).
    Quindi moltiplichiamo:
    \[ |B_2| = \binom{4}{2} \cdot \binom{6}{3} = 6 \cdot 20 = 120 \text{ casi favorevoli} \]

    \item \textbf{Calcolare la probabilità finale:}
    \[ \Prob(B_2) = \frac{\text{Casi favorevoli}}{\text{Casi possibili}} = \frac{120}{252} = \frac{10}{21} \approx 0.48 \]
\end{enumerate}

\textbf{Verifica della logica:}
\begin{itemize}
    \item Se il risultato fosse 1, vorrebbe dire che SEMPRE vanno 2 biondi a Roma (impossibile!)
    \item Se fosse 0, vorrebbe dire che è IMPOSSIBILE che vadano 2 biondi a Roma (anche questo impossibile!)
    \item 0.48 significa che circa nel 48\% dei casi, quando dividiamo la classe in modo casuale, finiranno esattamente 2 biondi a Roma
\end{itemize}
\end{example}

\section{Probabilità Condizionata}
\label{sec:prob_cond}
Spesso siamo interessati alla probabilità di un evento $A$ sapendo che un altro evento $B$ si è già verificato. Questa è la probabilità condizionata.

\begin{definition}[Probabilità Condizionata]
Siano $A$ e $B$ due eventi con $\Prob(B) > 0$. La \textbf{probabilità condizionata} di $A$ dato $B$, indicata con $\Prob(A|B)$, è definita come:
\[ \Prob(A|B) = \frac{\Prob(A \cap B)}{\Prob(B)} \]
Intuizione: $B$ diventa il "nuovo" spazio campionario. Stiamo valutando la probabilità della parte di $A$ che si trova in $B$, normalizzata rispetto alla probabilità di $B$.
\end{definition}

\begin{corollary}[Regola della Catena]\label{cor:catena}
Siano $A$ e $B$ due eventi con $\Prob(B) > 0$. Vale:
\[ \Prob(A \cap B) = \Prob(A|B) \Prob(B) \]
Se inoltre $\Prob(A) > 0$, vale anche:
\[ \Prob(A \cap B) = \Prob(B|A) \Prob(A) \]
Per tre eventi $A$, $B$, $C$:
\[ \Prob(A \cap B \cap C) = \Prob(A) \Prob(B|A) \Prob(C|A \cap B) \]
con generalizzazioni analoghe per $n$ eventi.
\end{corollary}

\begin{example}[Esercizio 1, 12/01/2024 - Urna di Polya]
\textit{Problema:} Urna con 6 palline (4 Rosse R, 2 Nere N). 3 estrazioni. Quando una pallina è estratta, viene reimbussolata con una nuova pallina dello stesso colore. Calcolare $\Prob(R_3)$, dove $R_3$ è "viene estratta una pallina rossa alla terza estrazione".
\textit{Soluzione (parte di essa, illustrando la regola della catena):}
Consideriamo una sequenza specifica, ad esempio $R_1 \cap N_2 \cap R_3$ (Rossa alla prima, Nera alla seconda, Rossa alla terza).
Usiamo la regola della catena: $\Prob(R_1 \cap N_2 \cap R_3) = \Prob(R_1) \Prob(N_2|R_1) \Prob(R_3|R_1 \cap N_2)$.
\begin{itemize}
    \item $\Prob(R_1)$: Inizialmente 4R, 2N (tot 6). $\Prob(R_1) = 4/6$.
    \item $\Prob(N_2|R_1)$: Se $R_1$ si è verificata, si è aggiunta una R. Urna ora: 5R, 2N (tot 7). $\Prob(N_2|R_1) = 2/7$.
    \item $\Prob(R_3|R_1 \cap N_2)$: Se $R_1$ e $N_2$ si sono verificate, si è aggiunta prima una R, poi una N. Urna ora: 5R, 3N (tot 8). $\Prob(R_3|R_1 \cap N_2) = 5/8$.
\end{itemize}
Quindi, $\Prob(R_1 \cap N_2 \cap R_3) = \frac{4}{6} \cdot \frac{2}{7} \cdot \frac{5}{8}$.
Per calcolare $\Prob(R_3)$ si devono considerare tutte le 8 possibili sequenze che portano a $R_3$ (RR$R_3$, RN$R_3$, NR$R_3$, NN$R_3$) e sommare le loro probabilità (lo vedremo meglio con la formula delle probabilità totali).
\end{example}

\subsection{Indipendenza tra Eventi}
\begin{definition}[Indipendenza tra Eventi]
Due eventi $A$ e $B$ si dicono \textbf{indipendenti} se il verificarsi di uno non influenza la probabilità del verificarsi dell'altro. Formalmente:
\[ \Prob(A \cap B) = \Prob(A) \Prob(B) \]
Se $\Prob(B)>0$, l'indipendenza è equivalente a $\Prob(A|B) = \Prob(A)$.
Se $\Prob(A)>0$, l'indipendenza è equivalente a $\Prob(B|A) = \Prob(B)$.
\end{definition}

\begin{remark}
Non confondere eventi indipendenti con eventi incompatibili!
\begin{itemize}
    \item Se $A, B$ sono incompatibili ($A \cap B = \emptyset$) e $\Prob(A)>0, \Prob(B)>0$, allora $\Prob(A \cap B) = 0$. Ma $\Prob(A)\Prob(B) > 0$. Quindi $A$ e $B$ \textbf{non} sono indipendenti (anzi, sono fortemente dipendenti: se uno si verifica, l'altro non può).
    \item L'unica eccezione è se uno degli eventi ha probabilità zero.
\end{itemize}
\end{remark}

\begin{example}
Lancio di due dadi. $A = \text{"il primo dado è 1"}$, $B = \text{"la somma dei dadi è 7"}$.
$\Omega$ ha $6 \times 6 = 36$ esiti equiprobabili.
$A = \{(1,1), (1,2), (1,3), (1,4), (1,5), (1,6)\}$, $\Prob(A) = 6/36 = 1/6$.
$B = \{(1,6), (2,5), (3,4), (4,3), (5,2), (6,1)\}$, $\Prob(B) = 6/36 = 1/6$.
$A \cap B = \{(1,6)\}$, $\Prob(A \cap B) = 1/36$.
Verifichiamo: $\Prob(A)\Prob(B) = (1/6)(1/6) = 1/36$.
Poiché $\Prob(A \cap B) = \Prob(A)\Prob(B)$, gli eventi $A$ e $B$ sono indipendenti.
\end{example}

\section{Formula delle Probabilità Totali e Teorema di Bayes}
Questi due strumenti sono potentissimi e molto usati negli esercizi.

\begin{definition}[Partizione dello Spazio Campionario]
Una famiglia di eventi $\{B_1, B_2, \dots, B_n\}$ costituisce una \textbf{partizione} di $\Omega$ se:
\begin{enumerate}
    \item $B_i \neq \emptyset$ per ogni $i$.
    \item $B_i \cap B_j = \emptyset$ per ogni $i \neq j$ (sono a due a due incompatibili).
    \item $\bigcup_{i=1}^n B_i = \Omega$ (la loro unione copre tutto lo spazio).
\end{enumerate}
\end{definition}

\begin{theorem}[Formula delle Probabilità Totali]
Sia $\{B_1, B_2, \dots, B_n\}$ una partizione di $\Omega$ tale che $\Prob(B_i) > 0$ per ogni $i$. Allora, per qualsiasi evento $A$:
\[ \Prob(A) = \sum_{i=1}^n \Prob(A \cap B_i) = \sum_{i=1}^n \Prob(A|B_i) \Prob(B_i) \]
\end{theorem}
\textit{Idea:} Per calcolare $\Prob(A)$, si "scompone" $A$ nelle sue parti che intersecano ciascun $B_i$, e si somma la probabilità di queste parti.

\begin{example}[Esercizio 1, 08/09/2023 - Gioco con due monete]
\textit{Problema (Parte 1):} Scommettitore sceglie a caso una di due monete: A (bilanciata, P(Testa)=1/2) o B (truccata, P(Testa)=2/3). Inizia con 1 euro, gioca max 5 volte. Vince 1 euro se Testa, perde 1 euro se Croce. Si ferma se perde tutto (0 euro) o vince 3 euro (ha 4 euro).
Calcolare $\Prob(I_2)$, dove $I_2 = \text{"il gioco si interrompe dopo il secondo lancio"}$.
\textit{Soluzione Passo Passo:}
\begin{enumerate}
    \item \textbf{Identificare la partizione:} L'evento iniziale che condiziona tutto è la scelta della moneta.
    Sia $D_A = \text{"viene scelta la moneta A"}$ e $D_B = \text{"viene scelta la moneta B"}$.
    Poiché la scelta è casuale, $\Prob(D_A) = 1/2$ e $\Prob(D_B) = 1/2$.
    $\{D_A, D_B\}$ è una partizione di $\Omega$ (relativo alla scelta della moneta).
    \item \textbf{Applicare la Formula delle Probabilità Totali a $I_2$:}
    $\Prob(I_2) = \Prob(I_2|D_A)\Prob(D_A) + \Prob(I_2|D_B)\Prob(D_B)$.
    \item \textbf{Analizzare $\Prob(I_2|D_A)$ e $\Prob(I_2|D_B)$:}
    Il giocatore inizia con 1 euro.
    Condizioni di stop: 0 euro o 4 euro.
    Perché il gioco si interrompa AL SECONDO lancio:
    \begin{itemize}
        \item NON deve interrompersi al primo lancio.
        \item DEVE interrompersi al secondo.
    \end{itemize}
    Analizziamo gli stati del capitale del giocatore ($S_0=1$):
    \begin{itemize}
        \item Lancio 1:
            \begin{itemize}
                \item Testa (T1): capitale $S_1=2$. Gioco continua.
                \item Croce (C1): capitale $S_1=0$. Gioco si ferma. Questo è $I_1$.
            \end{itemize}
    \end{itemize}
    Se il gioco arriva al secondo lancio, significa che al primo lancio è uscita Testa (capitale $S_1=2$).
    \begin{itemize}
        \item Lancio 2 (dato T1, quindi $S_1=2$):
            \begin{itemize}
                \item Testa (T2): capitale $S_2=3$. Gioco continua.
                \item Croce (C2): capitale $S_2=1$. Gioco continua.
            \end{itemize}
    \end{itemize}
    In nessuno dei due casi (Testa o Croce al secondo lancio) il giocatore ha 0 euro o 4 euro.
    Quindi, se il secondo lancio viene effettuato, il gioco NON si interrompe dopo il secondo lancio.
    L'unico modo per interrompersi \textbf{dopo} il secondo lancio sarebbe se al primo si fosse già interrotto, il che è una contraddizione.
    La soluzione fornita nell'esame afferma: "Si noti che, a prescindere dalla moneta utilizzata, se il secondo lancio viene effettuato, allora al primo lancio c'è stata una vittoria. In questo caso però, il giocatore dispone di 2 euro prima del secondo lancio, e perciò, se anche perdesse al secondo lancio, la sequenza non terminerebbe poiché avrebbe ancora un euro da giocare. Quindi $I_2 = \emptyset \Rightarrow \Prob(I_2) = 0$."
    Questo significa che $\Prob(I_2|D_A) = 0$ e $\Prob(I_2|D_B) = 0$.
    \item \textbf{Calcolare $\Prob(I_2)$:}
    $\Prob(I_2) = 0 \cdot (1/2) + 0 \cdot (1/2) = 0$.
\end{enumerate}
Questo esercizio specifico per $I_2$ era un po' un "trabocchetto" o un test di comprensione delle regole del gioco. Gli altri punti richiedono calcoli più standard.
\end{example}

\begin{theorem}[Teorema di Bayes]
Sia $\{B_1, B_2, \dots, B_n\}$ una partizione di $\Omega$ con $\Prob(B_i) > 0$ per ogni $i$. Sia $A$ un evento con $\Prob(A) > 0$. Allora, per ogni $k=1, \dots, n$:
\[ \Prob(B_k|A) = \frac{\Prob(A|B_k) \Prob(B_k)}{\Prob(A)} = \frac{\Prob(A|B_k) \Prob(B_k)}{\sum_{i=1}^n \Prob(A|B_i) \Prob(B_i)} \]
\end{theorem}
\textit{Idea:} Il teorema di Bayes ci permette di "invertire" la probabilità condizionata. Se conosciamo $\Prob(A|B_k)$ (la probabilità dell'effetto data la causa), possiamo calcolare $\Prob(B_k|A)$ (la probabilità della causa dato l'effetto).
$\Prob(B_k)$ è detta probabilità \textit{a priori} di $B_k$.
$\Prob(B_k|A)$ è detta probabilità \textit{a posteriori} di $B_k$ (dopo aver osservato $A$).

\begin{example}[Esercizio 1, Punto 4, 08/09/2023]
\textit{Problema:} Determinare $\Prob(D_A|P)$, dove $P = \text{"il gioco termina con lo scommettitore che non perde tutti i soldi"}$.
Ricordiamo $D_A = \text{"viene scelta la moneta A"}$.
\textit{Soluzione Passo Passo:}
\begin{enumerate}
    \item \textbf{Applicare il Teorema di Bayes:}
    \[ \Prob(D_A|P) = \frac{\Prob(P|D_A)\Prob(D_A)}{\Prob(P)} \]
    \item \textbf{Scomporre $\Prob(P)$ usando la Formula delle Probabilità Totali:}
    La partizione è $\{D_A, D_B\}$.
    \[ \Prob(P) = \Prob(P|D_A)\Prob(D_A) + \Prob(P|D_B)\Prob(D_B) \]
    Quindi:
    \[ \Prob(D_A|P) = \frac{\Prob(P|D_A)\Prob(D_A)}{\Prob(P|D_A)\Prob(D_A) + \Prob(P|D_B)\Prob(D_B)} \]
    \item \textbf{Utilizzare i valori forniti/calcolati nelle soluzioni dell'esame:}
    Sappiamo $\Prob(D_A) = 1/2$ e $\Prob(D_B) = 1/2$.
    La soluzione dell'esame al punto 3 calcola (anche se con un errore di battitura $D_A$ al posto di $D_B$ nella formula di $P(P^c|D_B)$) e poi fornisce:
    $\Prob(P|D_B) = 136/243$.
    $\Prob(P|D_A) = 5/16$. (Questo è calcolato nel testo della soluzione del punto 4).
    \item \textbf{Sostituire i valori:}
    \[ \Prob(D_A|P) = \frac{(5/16) \cdot (1/2)}{(5/16) \cdot (1/2) + (136/243) \cdot (1/2)} \]
    Si può semplificare il $1/2$:
    \[ \Prob(D_A|P) = \frac{5/16}{5/16 + 136/243} = \frac{5/16}{(5 \cdot 243 + 136 \cdot 16)/(16 \cdot 243)} = \frac{5/16}{(1215 + 2176)/3888} = \frac{5/16}{3391/3888} \]
    \[ = \frac{5}{16} \cdot \frac{3888}{3391} = \frac{5 \cdot 243}{3391} = \frac{1215}{3391} \]
    Questo corrisponde al risultato fornito nella soluzione dell'esame.
\end{enumerate}
\textbf{Come si calcolano $\Prob(P|D_A)$ e $\Prob(P|D_B)$?}
L'evento $P$ è "il gioco termina con lo scommettitore che non perde tutti i soldi". Questo significa che o il giocatore raggiunge i 4 euro, oppure arriva al 5° lancio senza essere andato in rovina e senza aver raggiunto i 4 euro, e termina con un capitale $>0$.
Alternativamente, è più facile calcolare $P^c = \text{"il giocatore perde tutti i soldi"}$, e poi fare $\Prob(P|D_X) = 1 - \Prob(P^c|D_X)$.
Per calcolare $\Prob(P^c|D_X)$ (probabilità di rovina dato che si usa la moneta X), si deve tracciare un albero delle possibili sequenze di gioco e dei capitali, fermandosi quando il capitale è 0 (rovina) o 4 (vittoria) o dopo 5 lanci.
Ad esempio, per $\Prob(P^c|D_A)$ (moneta bilanciata, $p=1/2$ per Testa, $q=1/2$ per Croce):
Capitale iniziale $S_0=1$.
Sequenze di rovina:
\begin{itemize}
    \item C (costo 1/2): $S_1=0$. Rovina. Prob = 1/2.
    \item TC C (costo 1/2, 1/2, 1/2): $S_0=1 \xrightarrow{T} S_1=2 \xrightarrow{C} S_2=1 \xrightarrow{C} S_3=0$. Rovina. Prob = $(1/2)^3 = 1/8$.
    \item TC TC C (costo $(1/2)^5$): $S_0=1 \xrightarrow{T} 2 \xrightarrow{C} 1 \xrightarrow{T} 2 \xrightarrow{C} 1 \xrightarrow{C} 0$. Rovina. Prob = $(1/2)^5 = 1/32$.
    \item TTT C C (costo $(1/2)^5$): $S_0=1 \xrightarrow{T} 2 \xrightarrow{T} 3 \xrightarrow{T} 4$ (STOP VITTORIA). Questa non è rovina.
    \item ... e altre.
\end{itemize}
La soluzione dell'esame usa un approccio che somma le probabilità dei percorsi che portano alla rovina, condizionati alla scelta della moneta.
Per $\Prob(P|D_A) = 1 - \Prob(P^c|D_A)$:
$\Prob(P^c|D_A)$ (rovina con moneta A, $p_A=1/2$):
\begin{itemize}
    \item C1: $1/2$
    \item T1 C2 C3: $(1/2)^3 = 1/8$
    \item T1 C2 T3 C4 C5: $(1/2)^5 = 1/32$
    \item T1 T2 C3 C4 C5: $(1/2)^5 = 1/32$
\end{itemize}
Somma delle probabilità di rovina: $1/2 + 1/8 + 1/32 + 1/32 = 16/32 + 4/32 + 1/32 + 1/32 = 22/32 = 11/16$.
Quindi $\Prob(P|D_A) = 1 - 11/16 = 5/16$. Questo è corretto come nella soluzione.

Per $\Prob(P^c|D_B)$ (rovina con moneta B, $p_B=2/3$ per Testa, $q_B=1/3$ per Croce):
\begin{itemize}
    \item C1: $q_B = 1/3$
    \item T1 C2 C3: $p_B q_B^2 = (2/3)(1/3)^2 = 2/27$
    \item T1 C2 T3 C4 C5: $p_B q_B p_B q_B^2 = p_B^2 q_B^3 = (2/3)^2 (1/3)^3 = 4/27 \cdot 1/27 = 4/243$ (La soluzione dell'esame ha un errore qui, usa $(1/3)^5$ che non ha senso, dovrebbe essere $p_B^2 q_B^3$. No, la soluzione dell'esame è P(C1|DB) + P(T1|DB)P(C2|DB/\text{T1})P(C3|DB/\text{T1}/\text{C2}) + ... che è corretto. Il mio errore era nell'interpretare la formula della soluzione.
    Rivediamo la soluzione dell'esame per $P(P^c|D_B)$:
    \begin{itemize}
        \item $P(C_1|D_B) = 1/3$ (Capitale 0. Rovina)
        \item $P(T_1 \cap C_2 \cap C_3 | D_B) = (2/3)(1/3)(1/3) = 2/27$ (Capitale $1 \to 2 \to 1 \to 0$. Rovina)
        \item $P(T_1 \cap C_2 \cap T_3 \cap C_4 \cap C_5 | D_B) = (2/3)(1/3)(2/3)(1/3)(1/3) = 4/243$ (Capitale $1 \to 2 \to 1 \to 2 \to 1 \to 0$. Rovina)
        \item $P(T_1 \cap T_2 \cap C_3 \cap C_4 \cap C_5 | D_B) = (2/3)(2/3)(1/3)(1/3)(1/3) = 4/243$ (Capitale $1 \to 2 \to 3 \to 2 \to 1 \to 0$. Rovina)
    \end{itemize}
    Sommando: $P(P^c|D_B) = 1/3 + 2/27 + 4/243 + 4/243 = (81+18+4+4)/243 = 107/243$.
    Quindi $\Prob(P|D_B) = 1 - 107/243 = (243-107)/243 = 136/243$. Questo è corretto.
\end{itemize}
La chiave è tracciare attentamente l'albero delle decisioni e dei capitali, fermandosi alle condizioni di stop (0 euro, 4 euro, o 5 lanci).
\end{example}

Questo conclude il primo capitolo sui fondamenti. Dovrebbe darti una base solida. Il prossimo passo sarà introdurre le variabili aleatorie.

% Continuo con la struttura LaTeX di base per i prossimi capitoli.
% Il contenuto dettagliato verrà poi espanso con più esempi dagli esami.

\end{document}
