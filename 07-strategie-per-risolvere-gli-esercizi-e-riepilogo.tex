\input{preambolo_comune}

% --- Titolo ---
\title{Strategie per Risolvere gli Esercizi e Riepilogo}
\author{Basato sulle prove d'esame}
\date{\today}

\begin{document}

\maketitle
\tableofcontents
\newpage

% ... contenuto ...
\begin{enumerate}
    \item \textbf{Leggi attentamente il testo}: Identifica cosa viene chiesto, quali sono i dati e quali sono le variabili aleatorie coinvolte.
    \item \textbf{Definisci lo Spazio Campionario $\Omega$ e gli Eventi di interesse}: Se possibile, conta gli elementi e verifica se lo spazio è equiprobabile.
    \item \textbf{Applica le formule base}: Unione, intersezione, complementare, probabilità condizionata.
    \item \textbf{Usa Formula delle Probabilità Totali e Bayes} quando hai una partizione dello spazio e probabilità condizionate.
    \item \textbf{Identifica le Variabili Aleatorie e le loro Distribuzioni}:
        \begin{itemize}
            \item È discreta o continua? Qual è il supporto?
            \item Riconosci una distribuzione nota (Binomiale, Poisson, Geometrica, Uniforme, Esponenziale, Normale)? Se sì, usa le sue proprietà (PMF/PDF, E[X], Var(X)).
            \item Se non è nota, calcola PMF/PDF, CDF, E[X], Var(X) dalle definizioni.
        \end{itemize}
    \item \textbf{Per Variabili Congiunte}:
        \begin{itemize}
            \item Scrivi la tabella (discreto) o la funzione (continuo) di probabilità/densità congiunta.
            \item Calcola marginali e condizionate se necessario.
            \item Verifica l'indipendenza.
            \item Calcola covarianza e correlazione.
        \end{itemize}
    \item \textbf{Per Catene di Markov}:
        \begin{itemize}
            \item Identifica gli stati e scrivi la matrice di transizione $P$. Disegna il grafo.
            \item Classifica gli stati (irriducibilità, aperiodicità, ricorrenza/transitorietà, assorbimento).
            \item Se richiesto, calcola $P^{(n)}$.
            \item Se la catena è irriducibile e aperiodica (e finita), calcola la distribuzione stazionaria $\pi$ risolvendo $\pi P = \pi$ e $\sum \pi_i = 1$.
            \item Se ci sono stati assorbenti, calcola le probabilità di assorbimento $h_i$.
        \end{itemize}
    \item \textbf{Controlla i risultati}: Le probabilità devono essere tra 0 e 1. Le varianze non negative. Le somme delle probabilità su tutto il supporto devono fare 1.
\end{enumerate}

\end{document}
