\input{preambolo_comune.tex}

% --- Titolo ---
\title{Calcolo Combinatorio}
\author{Alessandro Amella}
\date{\today}

\begin{document}

\maketitle
\tableofcontents
\newpage

\section{Probabilità Condizionata}

\subsection{Definizione e intuizione}

Spesso, nel calcolo delle probabilità, ci troviamo a valutare la probabilità di un evento $A$ quando siamo già a conoscenza che un altro evento $B$ si è verificato. Questa nuova informazione (il fatto che $B$ sia accaduto) può cambiare la nostra valutazione della probabilità di $A$.

\begin{definition}
Siano $A$ e $B$ due eventi, con $\Pc(B) > 0$. La \textbf{probabilità condizionata} (o condizionale) di $A$ dato $B$ è definita come:
\[ \Pcond{A}{B} = \frac{\Pc(A \capSet B)}{\Pc(B)} \]
\end{definition}

\begin{remark}[Intuizione sulla probabilità condizionata]
Sapere che $B$ si è verificato restringe il nostro "universo" dei possibili esiti. Non siamo più interessati a tutto $\OmegaSet$, ma solo agli esiti contenuti in $B$. All'interno di questo nuovo universo ridotto ($B$), vogliamo vedere qual è la "porzione" occupata dagli esiti che sono anche in $A$. Questa porzione è $A \capSet B$. La formula normalizza questa misura dividendola per la probabilità del nuovo universo, $\Pc(B)$, in modo che la somma delle probabilità degli esiti in $B$ (considerati nel nuovo contesto) sia 1.
\end{remark}

\begin{example}[Dado a 6 facce, probabilità uniforme]
Lanciamo un dado a 6 facce non truccato.
Lo spazio campionario è $\OmegaSet = \{1, 2, 3, 4, 5, 6\}$. Ogni esito ha probabilità $1/6$ (casi equiprobabili).

Sia $A = \text{"esce un numero maggiore o uguale a 3"} = \{3, 4, 5, 6\}$.
Sia $B = \text{"esce un numero pari"} = \{2, 4, 6\}$.
Vogliamo calcolare $\Pcond{A}{B}$, la probabilità che esca un numero $\ge 3$ sapendo che è uscito un numero pari.

\begin{itemize}
    \item L'evento $B$ ha probabilità $\Pc(B) = \Pc(\{2, 4, 6\}) = 3/6 = 1/2$.
    \item L'intersezione $A \capSet B$ (esiti che sono sia $\ge 3$ sia pari) è $\{4, 6\}$.
    \item La probabilità dell'intersezione è $\Pc(A \capSet B) = \Pc(\{4, 6\}) = 2/6 = 1/3$.
\end{itemize}
Applicando la definizione:
\[ \Pcond{A}{B} = \frac{\Pc(A \capSet B)}{\Pc(B)} = \frac{1/3}{1/2} = \frac{2}{3} \]
\begin{remark}[Verifica intuitiva]
Se sappiamo che è uscito un numero pari, i soli esiti possibili sono $\{2, 4, 6\}$ (questo è il nostro nuovo spazio campionario "ristretto", che ha 3 elementi). Di questi, quelli favorevoli ad $A$ (cioè $\ge 3$) sono $\{4, 6\}$ (2 elementi). Poiché, nel contesto ristretto di $B$, gli esiti $2, 4, 6$ sono ancora equiprobabili tra loro, la probabilità condizionata è $2/3$.
\end{remark}
\end{example}

\begin{example}[Dado truccato a 4 facce, probabilità non uniforme]
Lanciamo un dado truccato a 4 facce, $\OmegaSet = \{1, 2, 3, 4\}$.
Le probabilità dei singoli esiti sono:
$\Pc(\{1\}) = 8/15$, $\Pc(\{2\}) = 4/15$, $\Pc(\{3\}) = 2/15$, $\Pc(\{4\}) = 1/15$.
Sia $A = \text{"esce un numero maggiore o uguale a 3"} = \{3, 4\}$.
Sia $B = \text{"esce un numero pari"} = \{2, 4\}$.
Vogliamo calcolare $\Pcond{A}{B}$.

\begin{itemize}
    \item La probabilità dell'evento $B$ è $\Pc(B) = \Pc(\{2\}) + \Pc(\{4\}) = 4/15 + 1/15 = 5/15 = 1/3$.
    \item L'intersezione $A \capSet B$ (esiti $\ge 3$ e pari) è $\{4\}$.
    \item La probabilità dell'intersezione è $\Pc(A \capSet B) = \Pc(\{4\}) = 1/15$.
\end{itemize}
Applicando la definizione:
\[ \Pcond{A}{B} = \frac{\Pc(A \capSet B)}{\Pc(B)} = \frac{1/15}{1/3} = \frac{1}{5} \]
\begin{remark}[Approccio dei "veri" casi pesati]
Se sappiamo che è uscito un numero pari (evento $B$), i "veri" casi possibili sono 2 e 4. L'evento $A$ (numero $\ge 3$) si verifica solo se esce 4. Dobbiamo rapportare la probabilità del "vero" caso favorevole ($\Pc(\{4\})$) alla somma delle probabilità dei "veri" casi possibili ($\Pc(\{2\}) + \Pc(\{4\})$), che rappresenta la probabilità del nostro nuovo universo $\Pc(B)$:
\[ \Pcond{A}{B} = \frac{\Pc(\{4\})}{\Pc(\{2\}) + \Pc(\{4\})} = \frac{1/15}{4/15 + 1/15} = \frac{1/15}{5/15} = \frac{1}{5} \]
Questo conferma la formula generale. L'idea è che, anche se gli esiti originali non sono equiprobabili, una volta che ci si restringe allo spazio $B$, gli esiti in $B$ "ereditano" le loro probabilità originali, ma queste vengono "rinormalizzate" dividendo per $\Pc(B)$ in modo che la loro somma sia 1 nel nuovo contesto.
\end{remark}
\end{example}

\begin{exercise}
Considera l'esperimento del lancio di due dadi non truccati.
Sia $A = \text{"la somma dei punteggi è 7"}$.
Sia $B = \text{"il primo dado ha dato come risultato 3"}$.
Calcola $\Pcond{A}{B}$.
\end{exercise}
\begin{proof}[Soluzione]
Lo spazio campionario $\OmegaSet$ è l'insieme delle coppie ordinate $(d_1, d_2)$ dove $d_1, d_2 \in \{1, \dots, 6\}$ rappresentano l'esito del primo e del secondo dado, rispettivamente. Ci sono $6 \times 6 = 36$ esiti equiprobabili, ognuno con probabilità $1/36$. (Qui usiamo il principio fondamentale del calcolo combinatorio per le scelte indipendenti, o prodotto cartesiano degli insiemi degli esiti).
L'evento $B = \text{"il primo dado è 3"}$ è l'insieme di coppie $\{(3,1), (3,2), (3,3), (3,4), (3,5), (3,6)\}$.
Quindi, $\Pc(B) = |B|/|\OmegaSet| = 6/36 = 1/6$.
L'evento $A = \text{"la somma è 7"}$ è l'insieme di coppie $\{(1,6), (2,5), (3,4), (4,3), (5,2), (6,1)\}$.
L'evento $A \capSet B$ (somma 7 E primo dado 3) è l'esito (la coppia) $\{(3,4)\}$.
Quindi, $\Pc(A \capSet B) = 1/36$.
La probabilità condizionata è:
\[ \Pcond{A}{B} = \frac{\Pc(A \capSet B)}{\Pc(B)} = \frac{1/36}{1/6} = \frac{1}{36} \cdot 6 = \frac{1}{6} \]
\begin{remark}[Verifica intuitiva]
Se sappiamo che il primo dado è 3, affinché la somma sia 7, il secondo dado deve necessariamente dare 4. La probabilità che un dado non truccato (il secondo dado) dia 4 è $1/6$.
\end{remark}
\end{proof}

\subsection{Proprietà della probabilità condizionata}
Fissato un evento $B$ con $\Pc(B)>0$, la funzione $\Pcond{\cdot}{B}$ (che mappa un evento $A$ in $\Pcond{A}{B}$) è essa stessa una misura di probabilità. Ciò significa che soddisfa gli assiomi di Kolmogorov.

\begin{theorem}
Sia $B$ un evento tale che $\Pc(B)>0$. Valgono le seguenti proprietà:
\begin{enumerate}[label=\Roman*)]
    \item Per ciascun evento $A \subseteq \OmegaSet$, si ha $0 \le \Pcond{A}{B} \le 1$.
    \item $\Pcond{\OmegaSet}{B} = 1$.
    \item Se $A_1, A_2, \dots, A_n, \dots$ è una successione di eventi a due a due disgiunti, allora
    \[ \Pcond{\bigcup_{n=1}^\infty A_n}{B} = \sum_{n=1}^\infty \Pcond{A_n}{B} \quad (\sigma\text{-additività}) \]
    Da queste proprietà fondamentali derivano altre utili conseguenze:
    \item $\Pcond{\emptyset}{B} = 0$.
    \item Se $A_1, A_2$ sono disgiunti: $\Pcond{A_1 \cupSet A_2}{B} = \Pcond{A_1}{B} + \Pcond{A_2}{B}$ (additività finita).
    \item $\Pcond{\compSet{A}}{B} = 1 - \Pcond{A}{B}$.
    \item Se $A_1 \subseteq A_2$, allora $\Pcond{A_1}{B} \le \Pcond{A_2}{B}$ (monotonia).
\end{enumerate}
\end{theorem}
\begin{proof}[Dimostrazione (esempio per I)]
Dato che $A \capSet B \subseteq B$, per la monotonia della probabilità $\Pc$ (non condizionata), si ha $\Pc(A \capSet B) \le \Pc(B)$.
Poiché $\Pc(B)>0$ per ipotesi, possiamo dividere entrambi i membri per $\Pc(B)$ senza cambiare il verso della disuguaglianza:
\[ \frac{\Pc(A \capSet B)}{\Pc(B)} \le 1 \implies \Pcond{A}{B} \le 1 \]
Inoltre, la probabilità di un evento è sempre non negativa, quindi $\Pc(A \capSet B) \ge 0$. Dividendo per $\Pc(B) > 0$:
\[ \Pcond{A}{B} = \frac{\Pc(A \capSet B)}{\Pc(B)} \ge 0 \]
Mettendo insieme le due disuguaglianze, $0 \le \Pcond{A}{B} \le 1$. Le altre proprietà si dimostrano in modo analogo, usando la definizione di probabilità condizionata e le proprietà della misura di probabilità $\Pc$ non condizionata.
\end{proof}

\begin{remark}
È importante ricordare che, in generale, $\Pcond{A}{B} \neq \Pcond{B}{A}$.
Ad esempio, sia $A = \text{"una persona ha la febbre alta"}$ e $B = \text{"una persona ha l'influenza"}$.
$\Pcond{A}{B}$ (probabilità di febbre alta dato che ha l'influenza) è generalmente alta.
$\Pcond{B}{A}$ (probabilità di avere l'influenza dato che ha la febbre alta) potrebbe essere più bassa, perché la febbre alta può essere causata da molte altre malattie.
La relazione tra $\Pcond{A}{B}$ e $\Pcond{B}{A}$ è descritta dalla Formula di Bayes, che tratteremo in seguito.
\end{remark}


\section{Utilizzo della Probabilità Condizionata}

\subsection{Regola della catena (o formula della probabilità composta)}
La definizione di probabilità condizionata, $\Pcond{A}{B} = \frac{\Pc(A \capSet B)}{\Pc(B)}$, può essere riscritta per calcolare la probabilità dell'intersezione di due eventi. Se $\Pc(B)>0$, moltiplicando entrambi i membri per $\Pc(B)$ otteniamo:
\[ \Pc(A \capSet B) = \Pcond{A}{B} \Pc(B) \]
Questa è nota come \textbf{regola della catena} o formula della probabilità composta, ed è estremamente utile, specialmente in esperimenti che avvengono in più stadi (esperimenti sequenziali).

\begin{theorem}[Regola della catena per due eventi]
Siano $A$ e $B$ due eventi.
\begin{itemize}
    \item Se $\Pc(B) > 0$, allora: $\Pc(A \capSet B) = \Pcond{A}{B} \Pc(B)$.
    \item Se $\Pc(A) > 0$, allora: $\Pc(A \capSet B) = \Pcond{B}{A} \Pc(A)$.
\end{itemize}
\end{theorem}

Questa regola si può estendere a più di due eventi.
\begin{theorem}[Regola della catena per tre eventi]
Per tre eventi $A_1, A_2, A_3$, se $\Pc(A_1) > 0$ e $\Pc(A_1 \capSet A_2) > 0$ (condizioni che assicurano che le probabilità condizionate siano definite), allora:
\[ \Pc(A_1 \capSet A_2 \capSet A_3) = \Pcond{A_3}{A_1 \capSet A_2} \Pcond{A_2}{A_1} \Pc(A_1) \]
\end{theorem}
\begin{proof}[Idea della dimostrazione (per tre eventi)]
Partiamo da $\Pc(A_1 \capSet A_2 \capSet A_3)$. Possiamo vederla come $\Pc((A_1 \capSet A_2) \capSet A_3)$.
Sia $C = A_1 \capSet A_2$. Allora $\Pc(C \capSet A_3) = \Pcond{A_3}{C} \Pc(C)$ (usando la regola per due eventi).
Sostituendo $C$ con $A_1 \capSet A_2$, otteniamo $\Pcond{A_3}{A_1 \capSet A_2} \Pc(A_1 \capSet A_2)$.
Ora applichiamo di nuovo la regola della catena per due eventi a $\Pc(A_1 \capSet A_2)$:
$\Pc(A_1 \capSet A_2) = \Pcond{A_2}{A_1} \Pc(A_1)$.
Sostituendo quest'ultima espressione nella precedente, otteniamo la formula desiderata:
$\Pc(A_1 \capSet A_2 \capSet A_3) = \Pcond{A_3}{A_1 \capSet A_2} \Pcond{A_2}{A_1} \Pc(A_1)$.
\end{proof}

\begin{example}[Estrazioni da un'urna senza reimmissione]
Un'urna contiene 3 palline bianche (B), 2 palline nere (N) e 1 pallina rossa (R). In totale ci sono 6 palline. Si eseguono tre estrazioni \textbf{senza reimmissione} (ciò significa che la composizione dell'urna cambia dopo ogni estrazione).
Qual è la probabilità di estrarre nell'ordine: una bianca, poi una rossa, poi una nera?
Indichiamo con $B_1$ l'evento "prima estratta è bianca", $R_2$ l'evento "seconda estratta è rossa", e $N_3$ l'evento "terza estratta è nera". Vogliamo calcolare $\Pc(B_1 \capSet R_2 \capSet N_3)$.

Usiamo la regola della catena:
\[ \Pc(B_1 \capSet R_2 \capSet N_3) = \Pcond{N_3}{B_1 \capSet R_2} \Pcond{R_2}{B_1} \Pc(B_1) \]
Calcoliamo i termini uno per uno, tenendo traccia della composizione dell'urna:
\begin{itemize}
    \item $\Pc(B_1)$: All'inizio, l'urna ha (3B, 2N, 1R) - 6 palline totali. $\Pc(B_1) = \frac{\text{numero di Bianche}}{\text{totale palline}} = 3/6 = 1/2$.
    \item $\Pcond{R_2}{B_1}$: Se la prima pallina estratta è stata bianca ($B_1$ si è verificato), l'urna ora contiene (2B, 2N, 1R) - 5 palline totali. La probabilità di estrarre una rossa è $\Pcond{R_2}{B_1} = 1/5$.
    \item $\Pcond{N_3}{B_1 \capSet R_2}$: Se la prima è stata bianca e la seconda rossa ($B_1 \capSet R_2$ si è verificato), l'urna ora contiene (2B, 2N, 0R) - 4 palline totali. La probabilità di estrarre una nera è $\Pcond{N_3}{B_1 \capSet R_2} = 2/4 = 1/2$.
\end{itemize}
Mettendo insieme i pezzi:
\[ \Pc(B_1 \capSet R_2 \capSet N_3) = \left(\frac{3}{6}\right) \cdot \left(\frac{1}{5}\right) \cdot \left(\frac{2}{4}\right) = \frac{6}{120} = \frac{1}{20} \]
\end{example}

\subsection{Diagrammi ad albero}
I diagrammi ad albero sono uno strumento visuale molto efficace per organizzare e calcolare probabilità in esperimenti che si svolgono in più fasi.

\textbf{Caratteristiche di un diagramma ad albero:}
\begin{itemize}
    \item \textbf{Nodi:} Rappresentano gli stati o gli esiti parziali. La \textbf{radice} è $\OmegaSet$.
    \item \textbf{Rami:} Rappresentano i possibili esiti del sotto-esperimento successivo.
    \textbf{Probabilità sui rami:}
        \begin{itemize}
            \item Dal primo livello (dalla radice): probabilità \textit{non condizionate}.
            \item Dai livelli successivi: probabilità \textit{condizionate} agli eventi del cammino percorso fino a quel nodo.
        \end{itemize}
    \item \textbf{Probabilità di un cammino:} Si ottiene moltiplicando le probabilità sui rami del cammino (applicazione della regola della catena).
    \item \textbf{Somma delle probabilità dai nodi:} Deve fare 1 per i rami che partono da un medesimo nodo e rappresentano una partizione.
\end{itemize}

\begin{example}[Diagramma ad albero per l'estrazione B-R-N]
Urna: (3B, 2N, 1R). Tre estrazioni senza reimmissione. Calcolare $\Pc(B_1 \capSet R_2 \capSet N_3)$.

\begin{figure}[h!]
\centering
\begin{tikzpicture}[
    level distance=3cm,
    sibling distance=2cm,
    every node/.style={font=\small},
    state/.style={rectangle, draw, minimum width=1.8cm, minimum height=0.8cm},
    event/.style={midway, above, font=\scriptsize},
    prob/.style={midway, below, font=\scriptsize},
    path/.style={thick, -stealth},
    other/.style={gray, thin, -stealth}
]

% Nodo radice
\node[state] (root) {3B, 2N, 1R};

% Primo livello
\node[state, right=of root] (B1) {2B, 2N, 1R};
\draw[path] (root) -- (B1) node[event] {$B_1$} node[prob] {$3/6$};

\node[state, below=of B1] (N1) {3B, 1N, 1R};
\draw[other] (root) -- (N1) node[event] {$N_1$} node[prob] {$2/6$};

\node[state, below=of N1] (R1) {3B, 2N};
\draw[other] (root) -- (R1) node[event] {$R_1$} node[prob] {$1/6$};

% Secondo livello
\node[state, right=of B1] (R2) {2B, 2N};
\draw[path] (B1) -- (R2) node[event] {$R_2$} node[prob] {$1/5$};

\node[state, above=of R2] (B2) {1B, 2N, 1R};
\draw[other] (B1) -- (B2) node[event] {$B_2$} node[prob] {$2/5$};

\node[state, below=of R2] (N2) {2B, 1N, 1R};
\draw[other] (B1) -- (N2) node[event] {$N_2$} node[prob] {$2/5$};

% Terzo livello
\node[state, right=of R2] (N3) {2B, 1N};
\draw[path] (R2) -- (N3) node[event] {$N_3$} node[prob] {$2/4$};

\node[state, above=of N3] (B3) {1B, 2N};
\draw[other] (R2) -- (B3) node[event] {$B_3$} node[prob] {$2/4$};

% Probabilità finale
\node[right=of N3, text width=3cm] (prob) {
    $P(B_1 \capSet R_2 \capSet N_3)$
    $= \frac{3}{6} \cdot \frac{1}{5} \cdot \frac{2}{4}$
    $= \frac{1}{20}$
};
\draw[path, dashed] (N3) -- (prob);

\end{tikzpicture}
\caption{Diagramma ad albero per le estrazioni da un'urna senza reimmissione. Il cammino $B_1 \capSet R_2 \capSet N_3$ è evidenziato.}
\label{fig:tree_extraction_fixed}
\end{figure}

Il cammino che ci interessa è $\OmegaSet_0 \rightarrow B_1 \rightarrow R_2 \rightarrow N_3$.
La probabilità di questo cammino è:
$\Pc(B_1) \cdot \Pcond{R_2}{B_1} \cdot \Pcond{N_3}{B_1 \capSet R_2} = \frac{3}{6} \cdot \frac{1}{5} \cdot \frac{2}{4} = \frac{1}{20}$.
\end{example}

\subsection{Partizioni di \texorpdfstring{$\OmegaSet$}{Omega} (Schema di alternative)}
Una partizione divide lo spazio campionario in "fette" disgiunte la cui unione è l'intero spazio.

\begin{definition}
Una famiglia di eventi $H_1, H_2, \dots, H_n$ (o una famiglia numerabile $H_1, H_2, \dots$) forma una \textbf{partizione} di $\OmegaSet$ se:
\begin{enumerate}
    \item Gli eventi sono a due a due \textbf{disgiunti}: $H_i \capSet H_j = \emptyset$ per $i \neq j$.
    \item La loro \textbf{unione} è $\OmegaSet$: $\bigcup_{i} H_i = \OmegaSet$.
\end{enumerate}
\end{definition}

\begin{remark}
Se $\{H_1, H_2\}$ è una partizione, allora $H_2 = \compSet{H_1}$.
\end{remark}

\begin{example}[Scelta tra due urne e estrazione di una pallina]
\begin{itemize}
    \item Fase 1: Lancio di una moneta non truccata. Testa (T) $\implies$ Urna 1 (U1). Croce (C) $\implies$ Urna 2 (U2).
    \item Fase 2: Estrazione di una pallina dall'urna scelta.
\end{itemize}
U1: 2 Rosse (R), 1 Bianca (B). (Totale 3)
U2: 3 Rosse (R), 2 Bianche (B). (Totale 5)

Qual è la probabilità che esca Testa E la pallina estratta sia Bianca?
Eventi: $T = \text{"esce Testa"}$, $C = \text{"esce Croce"}$. $\{T,C\}$ è una partizione. $\Pc(T)=\Pc(C)=1/2$.
$B_{est} = \text{"estratta è Bianca"}$.
Vogliamo $\Pc(T \capSet B_{est})$.
Usiamo la regola della catena: $\Pc(T \capSet B_{est}) = \Pcond{B_{est}}{T} \Pc(T)$.
$\Pcond{B_{est}}{T}$ è la probabilità di estrarre Bianca da U1, che è $1/3$.
$\Pc(T \capSet B_{est}) = (1/3) \cdot (1/2) = 1/6$.
\end{example}

\begin{exercise}[Gioco con arresto, esempio semplificato]
Un giocatore inizia con 1 euro. Lancia una moneta bilanciata. T $\implies$ +1 euro, C $\implies$ -1 euro. Si ferma a 0 euro (rovina) o 3 euro (vittoria).
Probabilità di vittoria al secondo lancio?
\end{exercise}

\begin{proof}[Soluzione]
$S_k$ = capitale dopo $k$ lanci. $S_0 = 1$.
Vittoria al secondo lancio: $S_0=1 \rightarrow S_1 \in \{1,2\} \rightarrow S_2=3$.
\begin{itemize}
    \item Lancio 1:
        \begin{itemize}
            \item T (prob 1/2): $S_1 = 2$. Gioco continua.
            \item C (prob 1/2): $S_1 = 0$. Gioco fermo (rovina).
        \end{itemize}
    \item Lancio 2 (solo se $S_1=2$):
        \begin{itemize}
            \item T (prob 1/2): $S_2 = 3$. Gioco fermo (vittoria).
        \end{itemize}
\end{itemize}
Il cammino per "vittoria al secondo lancio" è $T_1 \capSet T_2$.
$\Pc(T_1 \capSet T_2) = \Pcond{T_2}{T_1} \Pc(T_1)$. Assumendo lanci indipendenti (concetto che vedremo meglio, ma qui intuitivo), $\Pcond{T_2}{T_1} = \Pc(T_2) = 1/2$.
$\Pc(\text{vittoria al secondo lancio}) = (1/2) \cdot (1/2) = 1/4$.
\end{proof}

\end{document}