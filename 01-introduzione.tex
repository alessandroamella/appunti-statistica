\input{preambolo_comune}

% --- Titolo ---
\title{Introduzione}
\author{Basato sulle prove d'esame}
\date{\today}

\begin{document}

\maketitle
\tableofcontents
\newpage

\section{A cosa serve la Probabilità e la Statistica}
Il Calcolo delle Probabilità è la branca della matematica che si occupa di analizzare e quantificare l'incertezza. Nella vita di tutti i giorni, così come in ambito scientifico, ingegneristico ed economico, ci troviamo costantemente di fronte a situazioni il cui esito non è predicibile con certezza. La probabilità ci fornisce gli strumenti per modellare queste situazioni, assegnare misure numeriche alla possibilità che certi eventi accadano e prendere decisioni informate in condizioni di incertezza.

La Statistica, d'altra parte, si occupa della raccolta, dell'analisi, dell'interpretazione e della presentazione dei dati. Spesso utilizza concetti probabilistici per trarre conclusioni su una popolazione più ampia a partire da un campione limitato di osservazioni (inferenza statistica) o per descrivere le caratteristiche principali di un insieme di dati (statistica descrittiva).

In questo corso, e in particolare in preparazione per l'esame, ci concentreremo sugli aspetti fondamentali del calcolo delle probabilità e su come applicarli per risolvere problemi specifici, come quelli visti nelle tracce d'esame.

\section{Come usare questa guida}
Questa guida è stata pensata per accompagnarti passo passo nello studio del Calcolo delle Probabilità e Statistica, con un focus specifico sulla preparazione per l'esame. Ogni capitolo tratterà un blocco di argomenti teorici, illustrandoli con definizioni precise, proprietà fondamentali e, soprattutto, esempi pratici.

\textbf{Approccio consigliato:}
\begin{enumerate}
    \item \textbf{Leggi la teoria attentamente:} Assicurati di aver compreso le definizioni e i concetti chiave prima di passare agli esempi.
    \item \textbf{Analizza gli esempi svolti:} Non limitarti a leggere la soluzione. Cerca di capire il ragionamento che porta a quella soluzione, identificando i concetti teorici applicati. Prova a rifare gli esempi da solo prima di guardare la soluzione.
    \item \textbf{Confronta con gli esercizi d'esame:} Molti esempi in questa guida saranno direttamente ispirati o tratti dagli esercizi d'esame che hai fornito. Questo ti aiuterà a familiarizzare con le tipologie di problemi e le tecniche risolutive richieste.
    \item \textbf{Fai pratica:} La matematica si impara facendola. Dopo aver studiato un capitolo, prova a risolvere esercizi simili, specialmente quelli delle tracce d'esame.
    \item \textbf{Non aver paura di tornare indietro:} Se un concetto non è chiaro, rileggi la parte teorica o gli esempi precedenti.
\end{enumerate}
L'obiettivo non è solo memorizzare formule, ma capire i principi che stanno dietro e saperli applicare in contesti diversi. In bocca al lupo!

\end{document}
