\input{preambolo_comune}

% --- Titolo ---
\title{Variabili Aleatorie Discrete}
\author{Basato sulle prove d'esame}
\date{\today}

\begin{document}

\maketitle
\tableofcontents
\newpage

\section{Introduzione}

\label{cap:va_discrete}
Una variabile aleatoria (v.a.) è una funzione che associa un numero reale a ogni esito di un esperimento aleatorio. Ci permette di passare da descrizioni qualitative a quantitative.

\begin{definition}[Variabile Aleatoria Discreta]
Una variabile aleatoria $X$ si dice \textbf{discreta} se l'insieme dei valori che può assumere (chiamato \textbf{supporto} $S_X$) è un insieme finito o numerabile (cioè i suoi elementi possono essere messi in corrispondenza biunivoca con i numeri naturali).
\end{definition}

Di seguito alcuni esempi generici e uno tratto da un esame:
\begin{itemize}
    \item $X = $ numero di Teste in 3 lanci di una moneta. $S_X = \{0, 1, 2, 3\}$.
    \item $Y = $ risultato del lancio di un dado. $S_Y = \{1, 2, 3, 4, 5, 6\}$.
    \item $Z = $ numero di tentativi fino al primo successo in una serie di prove. $S_Z = \{1, 2, 3, \dots\}$. (Numerabile infinito)
    \item Nell'Esercizio 3 dell'esame del 08/09/2023, si considera una v.a. $X$ con supporto $S_X = \{0, 2, 5\}$.
\end{itemize}

\section{Funzione di Massa di Probabilità (PMF)}
\begin{definition}[Funzione di Massa di Probabilità]
Per una v.a. discreta $X$, la \textbf{funzione di massa di probabilità} (PMF), indicata con $p_X(k)$ o $\Prob(X=k)$, assegna a ogni possibile valore $k$ del supporto $S_X$ la probabilità che $X$ assuma quel valore:
\[ p_X(k) = \Prob(X=k) \]
La PMF deve soddisfare:
\begin{enumerate}
    \item $p_X(k) \ge 0$ per ogni $k \in S_X$.
    \item $\sum_{k \in S_X} p_X(k) = 1$.
    \item $p_X(k) = 0$ se $k \notin S_X$.
\end{enumerate}
\end{definition}

\noindent\textbf{Esempio Semplice (PMF):} \\
Consideriamo il lancio di un dado equo a 6 facce. Sia $X$ il numero ottenuto.
\begin{itemize}
    \item Il supporto è $S_X = \{1, 2, 3, 4, 5, 6\}$
    \item La PMF è $p_X(k) = \frac{1}{6}$ per ogni $k \in S_X$
    \item Verifica: $\sum_{k=1}^6 p_X(k) = 6 \cdot \frac{1}{6} = 1$
    \item Se $k \notin S_X$ (ad esempio $k=7$), allora $p_X(k) = 0$
\end{itemize}

\section{Funzione di Ripartizione (CDF)}
\begin{definition}[Funzione di Ripartizione]
La \textbf{funzione di ripartizione} (CDF) di una v.a. $X$ (discreta o continua), indicata con $F_X(x)$, è definita per ogni $x \in \R$ come:
\[ F_X(x) = \Prob(X \le x) \]
Per una v.a. discreta, $F_X(x) = \sum_{k \le x, k \in S_X} p_X(k)$.

In sostanza, la funzione di ripartizione fornisce la probabilità che la variabile aleatoria assuma un valore minore o uguale a un dato valore $x$. Ad esempio, se $F_X(3) = 0.7$, significa che c'è una probabilità del 70\% che $X$ assuma un valore minore o uguale a 3.

Proprietà della CDF:
\begin{enumerate}
    \item $F_X(x)$ è crescente: se $x_2 > x_1$, allora $F_X(x_2) \ge F_X(x_1)$.
    \item $\lim_{x \to -\infty} F_X(x) = 0$.
    \item $\lim_{x \to +\infty} F_X(x) = 1$.
    \item $F_X(x)$ è continua da destra: $\lim_{h \to 0^+} F_X(x+h) = F_X(x)$.
    \item Per una v.a. discreta, $F_X(x)$ è una funzione a gradini. I salti avvengono nei punti $k \in S_X$ e l'altezza del salto in $k$ è $p_X(k)$.
    \item $\Prob(a < X \le b) = F_X(b) - F_X(a)$.
    \item $\Prob(X=k) = F_X(k) - F_X(k^-)$ (dove $F_X(k^-) = \lim_{x \to k^-} F_X(x)$).
\end{enumerate}
\end{definition}

\noindent\textbf{Esempio Semplice (CDF):} \\
Continuando l'esempio del dado equo, la CDF $F_X(x)$ è:
\[ F_X(x) = \begin{cases}
0 & \text{se } x < 1 \\
\frac{1}{6} & \text{se } 1 \le x < 2 \\
\frac{2}{6} & \text{se } 2 \le x < 3 \\
\frac{3}{6} & \text{se } 3 \le x < 4 \\
\frac{4}{6} & \text{se } 4 \le x < 5 \\
\frac{5}{6} & \text{se } 5 \le x < 6 \\
1 & \text{se } x \ge 6
\end{cases} \]

\noindent Vediamo alcuni esempi per capire meglio:

\begin{enumerate}
    \item \textbf{Esempio di valore continuo:} \\
    Quando scriviamo $F_X(3.5) = \frac{3}{6} = 0.5$, stiamo calcolando la probabilità di ottenere un numero minore o uguale a 3.5.
    Siccome i numeri possibili sono solo interi (1,2,3,4,5,6), dire ``minore o uguale a 3.5'' è equivalente a dire ``minore o uguale a 3''.
    Quindi $F_X(3.5)$ ci dà la probabilità di ottenere 1, 2 o 3, che è $\frac{3}{6} = 0.5$ (50\%).
    
    \item \textbf{Come calcolare la probabilità di ottenere 5 o 6:} \\
    Per trovare $P(X = 5 \text{ o } X = 6)$, usiamo la differenza $F_X(6) - F_X(4)$ perché:
    \begin{itemize}
        \item $F_X(6) = 1$ è la probabilità di ottenere un numero $\le 6$ (cioè qualsiasi numero)
        \item $F_X(4) = \frac{4}{6}$ è la probabilità di ottenere un numero $\le 4$ (cioè 1,2,3,4)
        \item La differenza $F_X(6) - F_X(4) = 1 - \frac{4}{6} = \frac{2}{6} = \frac{1}{3}$ ci dà quindi la probabilità di ottenere i numeri che sono maggiori di 4 ma minori o uguali a 6, cioè esattamente 5 e 6
    \end{itemize}
\end{enumerate}

\vspace{1em} 
\hrule
\par\vspace{0.5em}
\noindent\textbf{Esempio: Calcolo della Funzione di Ripartizione} (Tratto da: Esame 08/09/2023, Esercizio 3, Punto 1. Fa uso del valore atteso (def. \ref{def:valore-atteso}) e della varianza (def. \ref{def:varianza}))
\par\vspace{0.5em}
\noindent\textbf{Problema:}
\begin{quote}
\itshape
Sia X una variabile aleatoria tale che $S_X = \{0,2,5\}$, $E[X] = 2 = \text{var}(X)$. Si determini:
1) la funzione di ripartizione di X.
\end{quote}
\noindent\textbf{Soluzione:}
\begin{quote}
\textit{Soluzione Passo Passo:}
\begin{enumerate}
    \item \textbf{Impostare il sistema di equazioni:}
    Siano $p_0 = \Prob(X=0)$, $p_2 = \Prob(X=2)$, $p_5 = \Prob(X=5)$.
    Sappiamo che:
    \begin{itemize}
        \item $p_0 + p_2 + p_5 = 1$ (la somma delle probabilità è 1).
        \item $E[X] = 0 \cdot p_0 + 2 \cdot p_2 + 5 \cdot p_5 = 2p_2 + 5p_5 = 2$.
        \item $\Var(X) = E[X^2] - (E[X])^2 = 2$.
          $E[X^2] = 0^2 p_0 + 2^2 p_2 + 5^2 p_5 = 4p_2 + 25p_5$.
          Quindi, $4p_2 + 25p_5 - (2)^2 = 2 \Rightarrow 4p_2 + 25p_5 - 4 = 2 \Rightarrow 4p_2 + 25p_5 = 6$.
    \end{itemize}
    Abbiamo il sistema:
    \begin{align*}
        p_0 + p_2 + p_5 &= 1 \\
        2p_2 + 5p_5 &= 2 \\
        4p_2 + 25p_5 &= 6
    \end{align*}
    \item \textbf{Risolvere il sistema per $p_2, p_5$ (ultime due equazioni):}
    Dalla seconda: $2p_2 = 2 - 5p_5 \Rightarrow p_2 = 1 - \frac{5}{2}p_5$.
    Sostituisci nella terza: $4(1 - \frac{5}{2}p_5) + 25p_5 = 6$
    $4 - 10p_5 + 25p_5 = 6 \Rightarrow 15p_5 = 2 \Rightarrow p_5 = 2/15$.
    Sostituisci $p_5$ per trovare $p_2$: $p_2 = 1 - \frac{5}{2}(\frac{2}{15}) = 1 - \frac{1}{3} = 2/3$.
    \item \textbf{Trovare $p_0$ dalla prima equazione:}
    $p_0 + 2/3 + 2/15 = 1 \Rightarrow p_0 = 1 - 2/3 - 2/15 = 1 - 10/15 - 2/15 = 1 - 12/15 = 3/15 = 1/5$.
    Quindi, la PMF è: $\Prob(X=0)=1/5$, $\Prob(X=2)=2/3$, $\Prob(X=5)=2/15$.
    Verifica: $1/5 + 2/3 + 2/15 = (3+10+2)/15 = 15/15=1$. Corretto.
    \item \textbf{Scrivere la Funzione di Ripartizione $F_X(x)$:}
    $F_X(x)$ è una funzione a gradini:
    \begin{itemize}
        \item Se $x < 0$, $F_X(x) = \Prob(X \le x) = 0$.
        \item Se $0 \le x < 2$, $F_X(x) = \Prob(X=0) = 1/5$.
        \item Se $2 \le x < 5$, $F_X(x) = \Prob(X=0) + \Prob(X=2) = 1/5 + 2/3 = 3/15 + 10/15 = 13/15$.
        \item Se $x \ge 5$, $F_X(x) = \Prob(X=0) + \Prob(X=2) + \Prob(X=5) = 1/5 + 2/3 + 2/15 = 1$.
    \end{itemize}
    Quindi:
    \[ F_X(x) = \begin{cases} 0 & \text{se } x < 0 \\ 1/5 & \text{se } 0 \le x < 2 \\ 13/15 & \text{se } 2 \le x < 5 \\ 1 & \text{se } x \ge 5 \end{cases} \]
\end{enumerate}
\end{quote}
\vspace{0.5em}
\hrule
\vspace{1em}


\section{Valore Atteso (Media) e Varianza}
\begin{definition}[Valore Atteso]\label{def:valore-atteso}
Il \textbf{valore atteso} (o media) di una v.a. discreta $X$, indicato con $\E[X]$ (o $\mu_X$), è:
\[ \E[X] = \sum_{k \in S_X} k \cdot p_X(k) \]
Rappresenta il valore medio che ci si aspetta di osservare per $X$ su un gran numero di ripetizioni dell'esperimento.
Se $g(X)$ è una funzione di $X$, allora $\E[g(X)] = \sum_{k \in S_X} g(k) \cdot p_X(k)$.
Proprietà (Linearità):
\begin{itemize}
    \item $\E[aX+b] = a\E[X]+b$ per costanti $a,b$.
    \item $\E[X+Y] = \E[X]+\E[Y]$ (\underline{anche se $X,Y$ non sono indipendenti}).
\end{itemize}
\end{definition}

\begin{definition}[Varianza]\label{def:varianza}
La \textbf{varianza} di una v.a. $X$, indicata con $\Var(X)$ (o $\sigma_X^2$), misura la dispersione dei valori di $X$ attorno alla sua media $\E[X]$:
\[ \Var(X) = \E[(X - \E[X])^2] = \sum_{k \in S_X} (k - \E[X])^2 p_X(k) \]
Una formula computazionalmente più comoda è:
\[ \boxed{\Var(X) = \E[X^2] - (\E[X])^2} \]
dove $\E[X^2] = \sum_{k \in S_X} k^2 p_X(k)$.
Proprietà:
\begin{itemize}
    \item $\Var(X) \ge 0$.
    \item $\Var(aX+b) = a^2 \Var(X)$ per costanti $a,b$.
    \item Se $X, Y$ \underline{sono indipendenti}, $\Var(X+Y) = \Var(X) + \Var(Y)$.
\end{itemize}
La \textbf{deviazione standard} $\sigma_X = \sqrt{\Var(X)}$ ha la stessa unità di misura di $X$.
\end{definition}

\section{Principali Distribuzioni Discrete}
Vediamo le distribuzioni che appaiono più frequentemente negli esercizi.

\subsection{Distribuzione di Bernoulli \texorpdfstring{$X \sim \text{Be}(p)$}{X ~ Be(p)}}
Descrive un esperimento con due soli esiti: "successo" (valore 1) e "insuccesso" (valore 0).

Il parametro $p$ rappresenta la probabilità di successo dell'esperimento (es. per una moneta equa, $p=0.5$).

$S_X = \{0, 1\}$

\textbf{PMF:}
\begin{itemize}
    \item $\Prob(X=1) = p$
    \item $\Prob(X=0) = 1-p = q$
\end{itemize}

\textbf{Valore atteso:} $\E[X] = p$

\textbf{Varianza:} $\Var(X) = p(1-p) = pq$

\subsection{Distribuzione Binomiale \texorpdfstring{$X \sim \text{Bin}(n,p)$}{X ~ Bin(n,p)}}
Descrive il numero di successi in $n$ prove di Bernoulli indipendenti, ognuna con probabilità di successo $p$.

$S_X = \{0, 1, \dots, n\}$

\textbf{PMF:} 
\[ \Prob(X=k) = \binom{n}{k} p^k (1-p)^{n-k} \quad \text{per } k \in S_X \]

\textbf{Valore atteso:} $\E[X] = np$

\textbf{Varianza:} $\Var(X) = np(1-p) = npq$

\vspace{1em}
\hrule
\par\vspace{0.5em}
\noindent\textbf{Esempio: Numero di palline rosse} (Tratto da: Esame 19/06/2024, Esercizio 3, Punto a)
\par\vspace{0.5em}
\noindent\textbf{Problema:}
\begin{quote}
\itshape
Fissiamo $p \in (0, 1)$ e consideriamo una moneta che dà testa con probabilità $p$. Un'urna inizialmente è vuota. Lancio una moneta: se esce testa, inserisco nell'urna una pallina rossa, altrimenti ne inserisco una verde. Lancio ancora la moneta ed inserisco una seconda pallina nell'urna (rossa se esce testa, verde altrimenti). Itero questa procedura $n$ volte, ottenendo un'urna che contiene $n$ palline. Indichiamo con $X$ il numero di palline rosse nell'urna.
Determinare la probabilità che l'urna contenga $k$ palline rosse, per ogni $k \in \{0, 1, \dots, n\}$.
\end{quote}
\noindent\textbf{Soluzione:}
\begin{quote}
Ogni lancio della moneta è una prova di Bernoulli: "successo" (pallina rossa) se esce Testa (con probabilità $p$), "insuccesso" (pallina verde) se esce Croce (con probabilità $1-p$). Le $n$ prove (lanci) sono indipendenti.
La variabile aleatoria $X$, che conta il numero di palline rosse (successi) in $n$ prove, segue una distribuzione Binomiale con parametri $n$ (numero di prove) e $p$ (probabilità di successo).
La sua PMF è:
\[ \Prob(X=k) = \binom{n}{k} p^k (1-p)^{n-k}, \quad \text{per } k \in \{0, 1, \dots, n\}. \]
\end{quote}
\vspace{0.5em}
\hrule
\vspace{1em}


\subsection{Distribuzione Geometrica \texorpdfstring{$X \sim \text{Geo}(p)$}{X ~ Geo(p)}}
Descrive il numero di prove di Bernoulli indipendenti necessarie per ottenere il \textbf{primo} successo.

$S_X = \{1, 2, 3, \dots\}$

\textbf{PMF:} 
\[ \Prob(X=k) = (1-p)^{k-1} p = q^{k-1}p \quad \text{per } k \in S_X \]

\textbf{Valore atteso:} $\E[X] = 1/p$

\textbf{Varianza:} $\Var(X) = (1-p)/p^2 = q/p^2$

\textbf{Proprietà di assenza di memoria} (per la versione che conta il numero di prove):
\[ \Prob(X > k+j | X > j) = \Prob(X > k) \]

\textbf{CDF:} 
\[ F_X(k) = \Prob(X \le k) = 1 - (1-p)^k = 1 - q^k \]

\vspace{1em}
\hrule
\par\vspace{0.5em}
\noindent\textbf{Esempio: CDF della Distribuzione Geometrica} (Tratto da: Esame 13/09/2024, Esercizio 3, Punto a)
\par\vspace{0.5em}
\noindent\textbf{Problema:}
\begin{quote}
\itshape
Sia $X$ una v.a. di legge geometrica di parametro $p=1/2$, ovvero con densità discreta $p_X(k) = (1/2)(1-1/2)^{k-1} = (1/2)^k$, per $k \in \mathbb{N} = \{1,2,3,\dots\}$.
Mostrare che $X$ ha funzione di ripartizione $F_X(k) = 1 - (1/2)^k$, per $k \in \mathbb{N}$.
\end{quote}
\noindent\textbf{Soluzione:}
\begin{quote}
La funzione di ripartizione $F_X(k)$ è definita come $P(X \le k)$. Per $k \in \mathbb{N}$:
\[ F_X(k) = \sum_{j=1}^{k} P(X=j) = \sum_{j=1}^{k} (1/2)^j \]
Questa è una somma parziale di una serie geometrica con primo termine $a = 1/2$, ragione $r = 1/2$, e $k$ termini.
La formula per la somma parziale di una serie geometrica è $S_k = a \frac{1-r^k}{1-r}$.
Applicandola:
\[ F_X(k) = (1/2) \frac{1-(1/2)^k}{1-(1/2)} = (1/2) \frac{1-(1/2)^k}{1/2} = 1-(1/2)^k \]
Questo dimostra quanto richiesto.
\end{quote}
\vspace{0.5em}
\hrule
\vspace{1em}

\noindent \textit{Nota contestuale:} Nell'Esercizio 2 dell'esame del 17/07/2024 (Gioco di Agata), la distanza $T$ della bandierina è definita come una variabile aleatoria geometrica di parametro $p=1/2$, con densità discreta $p_T(k) = (1/2)^k$ per $k \in \mathbb{N} = \{1, 2, \dots\}$. Questo è un esempio di come la distribuzione geometrica può apparire in un problema.


\subsection{Distribuzione di Poisson \texorpdfstring{$X \sim \text{Po}(\lambda)$}{X ~ Po(lambda)}}
Descrive il numero di eventi che si verificano in un intervallo di tempo o spazio fissato, quando questi eventi accadono con una frequenza media nota $\lambda$ e indipendentemente l'uno dall'altro.

$S_X = \{0, 1, 2, \dots\}$

\textbf{PMF:} 
\[ \Prob(X=k) = e^{-\lambda} \frac{\lambda^k}{k!} \quad \text{per } k \in S_X \]

\textbf{Valore atteso:} $\E[X] = \lambda$

\textbf{Varianza:} $\Var(X) = \lambda$

\textbf{Nota:} La Binomiale $\text{Bin}(n,p)$ può essere approssimata da una Poisson $\text{Po}(\lambda=np)$ se $n$ è grande e $p$ è piccolo.

\vspace{1em}
\hrule
\par\vspace{0.5em}
\noindent\textbf{Esempio: Covarianza con una v.a. di Poisson} (Tratto da: Esame 08/09/2023, Esercizio 3, Punto 2)
\par\vspace{0.5em}
\noindent\textbf{Problema:}
\begin{quote}
\itshape
Sia X la variabile aleatoria definita nel punto 1 ($S_X = \{0,2,5\}$, $E[X] = 2$, $\text{var}(X)=2$).
Si determini:
2) $\text{cov}(2X + Z, -X +3Z)$ dove Z è una variabile aleatoria di Poisson di parametro $\lambda = 1$ indipendente da X.
\end{quote}
\noindent\textbf{Soluzione:}
\begin{quote}
\textit{Soluzione Passo Passo:}

\textbf{1. Ricordiamo la bilinearità della covarianza:}
Per variabili aleatorie $A,B,C,D$ e costanti $a,b,c,d$:
\[ \Cov(aA+bB, cC+dD) = ac \Cov(A,C) + ad \Cov(A,D) + bc \Cov(B,C) + bd \Cov(B,D) \]
Nel nostro caso, $A=X, B=Z, C=X, D=Z$.
\[ \Cov(2X+Z, -X+3Z) \]

\textbf{2. Applichiamo la formula:}
\begin{align*}
    \Cov(2X+Z, -X+3Z) &= \Cov(2X, -X) + \Cov(2X, 3Z) + \Cov(Z, -X) + \Cov(Z, 3Z) \\
    &= 2(-1)\Cov(X,X) + 2(3)\Cov(X,Z) + 1(-1)\Cov(Z,X) + 1(3)\Cov(Z,Z) \\
    &= -2 \Cov(X,X) + 6 \Cov(X,Z) - \Cov(Z,X) + 3 \Cov(Z,Z)
\end{align*}

\textbf{3. Ricordiamo che:}
\begin{itemize}
    \item $\Cov(A,A) = \Var(A)$ 
    \item $\Cov(A,B) = \Cov(B,A)$
\end{itemize}

\textbf{4. Quindi:}
\begin{align*}
    \text{L'espressione diventa } &= -2 \Var(X) + (6-1) \Cov(X,Z) + 3 \Var(Z) \\
    &= -2 \Var(X) + 5 \Cov(X,Z) + 3 \Var(Z)
\end{align*}

\textbf{5. Utilizziamo i dati del problema:}
\begin{itemize}
    \item $\Var(X) = 2$ (dato dal punto 1 dell'esercizio).
    \item $Z \sim \text{Po}(1)$, quindi per una Poisson $\E[Z]=\lambda=1$ e $\Var(Z)=\lambda=1$.
    \item $X$ e $Z$ sono indipendenti. Se due v.a. sono indipendenti, la loro covarianza è 0. Quindi $\Cov(X,Z)=0$.
\end{itemize}

\textbf{6. Sostituendo i valori:}
\[ \Cov(2X+Z, -X+3Z) = -2(2) + 5(0) + 3(1) = -4 + 0 + 3 = -1 \]
\end{quote}
\vspace{0.5em}
\hrule
\vspace{1em}


\subsection{Distribuzione Uniforme Discreta \texorpdfstring{$X \sim \text{Unif}\{x_1, \dots, x_n\}$}{X ~ Unif}}
Una v.a. $X$ assume $n$ valori distinti $\{x_1, \dots, x_n\}$ ciascuno con la stessa probabilità $1/n$.

$S_X = \{x_1, \dots, x_n\}$

\textbf{PMF:} 
\[ \Prob(X=x_i) = \frac{1}{n} \quad \text{per } i=1, \dots, n \]

\textbf{Se $x_i = i$ per $i=1, \dots, n$:}
\begin{itemize}
    \item \textbf{Valore atteso:} $\E[X] = \frac{n+1}{2}$
    \item \textbf{Varianza:} $\Var(X) = \frac{n^2-1}{12}$
\end{itemize}

\noindent \textit{Nota contestuale:} Nell'Esercizio 1 dell'esame del 17/07/2024, si considerava il lancio di due dadi regolari a sei facce, con esiti $X$ e $Y$. In tale contesto, $X$ e $Y$ sono variabili aleatorie indipendenti che seguono una distribuzione uniforme discreta sull'insieme $\{1, 2, 3, 4, 5, 6\}$, cioè $X, Y \sim \text{Unif}\{1, 2, 3, 4, 5, 6\}$. La probabilità $P(X=k)=1/6$ per $k \in \{1, \dots, 6\}$, e analogamente per $Y$.

\end{document}